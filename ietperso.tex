\documentclass{ietperso}

%TC:envir table [] ignore
%TC:envir figure [] ignore
\begin{document}
\title{Unofficial Submission Template for IET Research Journal Papers}
\author[1]{First Author}
\author[2,*]{Second Author}
\author[3,4]{Third Author}
\affil[1]{First Department, First University, Address, City, Country Name}
\affil[2]{Second Company Department, Company Address, City, Country Name}
\affil[3]{Third Department, Third University, Address, Country Name}
\affil[4]{Current affiliation: Fourth Department, Fourth University, Address, Country Name}
\affil[*]{\url{corresponding.author@second.com}}

\maketitle

%============================================ ABSTRACT ============================================%
%TC:break abstract
\begin{ietabstract}
This should be informative and suitable for direct inclusion in abstracting services as a self-contained article. It should not exceed 200 words. It should summarise the general scope and also state the main results obtained, methods used, the value of the work and the conclusions drawn. No figure numbers, table numbers, references or displayed mathematical expressions should be included. The abstract should be included in both the Manuscript Central submission step (Step 1) and the submitted paper.
\end{ietabstract}

%============================================== BODY ==============================================%
\begin{ietbody}
\section{Introduction}
This document is a unofficial template for submission to the IET Research Journal Papers. Note this template is unofficial. There is no guarantee your article won't be rejected by the editors desk for this very reason. As such, use at your own risks.

Before submitting your final paper, check that the format conforms to the Author Guide~\cite{author_2015_guide}.  Specifically, check to make sure that the correct referencing style has been used and the citations are in numerical order throughout the text. If your paper does not meet all of the requirements, your paper will be unsubmitted and you will be asked to correct it.

\section{Language, spelling and grammar}
All papers must be written in UK English. If English is not your first language, you should ask an English-speaking colleague to proofread your paper. Papers that fail to meet basic standards of literacy are likely to be unsubmitted by the Editorial Office.

\section{Length}
Original research papers submitted to the IET Research Journals should conform to the IET Research Journals Length Policy~\cite{author_2015_length}. The length guidelines include the abstract, references and appendices but do not include figure captions or table content.

\section{Author names and affiliations}
Names and affiliations should immediately follow the title. To avoid confusion, the family name must be written as the last part of each author name and the first name should be spelt out rather than abbreviated (e.g. John A.K. Smith). Author details must not show any professional title (e.g. Managing Director), any academic title (e.g. Dr.) or any membership of any professional organisation.).

For multiple-authored articles, list the full names of all the authors, using identifiers to link an author with an affiliation where necessary (eg. John AK Smith$^1$, Edward Jones$^2$).

The full affiliations of all authors should then be listed. Affiliations should include: the department name; the name of the university or company; the name of the city; and the name of the country (e.g. $^1$Department of Electrical Engineering, University of Sydney, Sydney, Australia).

If an author's present address is different from the address at which the work was carried out, this should be given as a secondary affiliation (see affiliation 4).

Only the email address of the corresponding author is required and should be indicated with a *. 
All co-authors must be listed on ScholarOne Manuscript Central as part of the submission process. There is also the opportunity to include ORCID  IDs for all authors in step 3 of the submission steps~\cite{author_2015_orcid}. If you do not know a co-author's ORCID ID there is a look-up option included in Manuscript Central.

\section{Page formatting}
An easy way to comply with the requirements stated in the Author Guide~\cite{author_2015_guide} is to use this document as a template and simply type your text into it. PDF files are also accepted, so long as they follow the same style. Authors should not copy the format of the published journal; all accepted papers will be edited into the IET Research Journals house-style during the production process. 

\subsection{Page Layout}
Your paper must be in single column format with double spacing to ensure that reviewers can easily read and mark up the paper if required.

All paragraphs must be justified, i.e. both left-justified and right-justified.

\subsection{Text Font of Entire Document}
It is recommended that standardised fonts such as Times New Roman and Arial are used with a font size no smaller than 10pt.

\subsection{Section Headings}
All section heading should be numbered and no more than 3 levels of headings should be used.

\subsubsection{First Level Headings}
The first level section headings should be in bold font (e.g. ``1. Introduction''), with the paragraph starting on a new line.

\subsubsection{Second Level Headings}
The second level section headings should be in italic font (i.e. ``2.3 Section Headings'').  The paragraph should start on a new line.

\subsubsection{Third Level Headings}
The third level section headings should also be in italic font but should end with a colon (:). The text for that section should run on and not start as a new paragraph.

\section{Figures}
Figures will be reproduced exactly as supplied, with no redrawing or relabelling. It is therefore imperative that the supplied figures are of the highest possible quality. The preferred format is encapsulated postscript (.eps) for line figures and .tif for halftone figures with a minimum resolution of 300 dpi (dots per inch)

Graphics may be full colour but please make sure that they are appropriate for print (black and white) and online (colour) publication. For example lines graphs should be colour and use dotted or dashed lines, or shapes to distinguish them apart in print. For example, see Fig.~\ref{fig:1}. Each figure should be explicitly referred to in numerical order and should be embedded within the text at the appropriate point. A maximum of four subfigures will be allowed per figure.

\begin{figure}
\centering
\psfragfig[width=8cm]{fig}
\caption{\label{fig:1}Sample graph}
\end{figure}

\section{Tables}
Tables should be formatted as the example below with no column lines unless needed to clarify the content of the table. Row lines can be used to distinguish the column headings from the content of the table.

\begin{table}
\caption{\label{tab:table1}Example table}
\begin{tabular}{lcc}
\hline
Column heading & Column heading two & Column heading three \\
\hline
Row 1a & Row 1b & Row 1c \\
Row 2a & Row 2b & Row 2c \\
\hline
\end{tabular}
\end{table}

\subsection{Table Captions}
Tables must be numbered and cited within the text in strict numerical order. Table captions must be above the table and in 10pt font.

\section{Mathematics and equations}
When writing mathematics, avoid confusion between characters that could be mistaken for one another, e.g. the letter 'l' and the number one.

Equations should be capable of fitting into a two-column print format.

Vectors and matrices should be in bold italic and variables in italic.

If your paper contains superscripts or subscripts, take special care to ensure that the positioning of the characters is unambiguous.

Exponential expressions should be written using superscript notation, i.e. $5\times 10^3$ not $5\text{E}03$. A multiplication sign should be used, not a dot.
Refer to equations using round brackets, e.g.~\eqref{eq:1}

\begin{equation}\label{eq:1}
ax = b
\end{equation}

\section{Page Numbers and Footers}
Page numbers should be used on all pages. Footers are not in IET’s house style so must not be used. If they are used they will be moved to within the text at the typesetting stage.

\section{Conclusion}
Submissions should always include the following sections: an abstract; an introduction; a conclusion and a references section. If any of the above sections are not included the paper will be unsubmitted and you will be asked to add the relevant section. 

\section{Acknowledgements}
Acknowledgements should be placed after the conclusion and before the references section. This is where reference to any grant numbers or supporting bodies should be included. The funding information should also be entered into the first submission step on Manuscript Central which collects Fundref information~\cite{author_2015_fundref}.

\section{Biographies}
Please note that the IET does not include author biographies in published papers.

\section{References}
An average research paper should reference between 20 and 30 works, the bulk of which should be recently published (i.e. within the last five years) leading-edge articles in the field, preferably from top journals or conferences. You should compare your own findings to this recent research and demonstrate how your work improves on it in order to demonstrate that your work shows a significant advance over the state of the art – a pre-requisite for publication in IET Research Journals.

\subsection{Referencing Style}
You should number your references sequentially throughout the text, and each reference should be individually numbered and enclosed in square brackets (e.g.~\cite{author_2015_guide}).

Please ensure that all references in the Reference list are cited in the text and vice versa. Failure to do so may cause delays in the production of your article.

Please also ensure that you provide as much information as possible to allow the reader to locate the article concerned. This is particularly important for articles appearing in conferences, workshops and books that may not appear in journal databases.

Do not include references for papers that have been submitted and not accepted for publication. Papers that have been accepted for publication are allowed as long as all information is provided.

Please provide all author name(s) and initials, title of the paper, date published, title of the journal or book, volume number, editors (if any), and finally the page range. For books and conferences, the town of publication and publisher (in parentheses) should also be given.

If the number of authors on a reference is greater than three please list the first three authors followed by \textit{et al}. 

\section{Appendices}
Additional material, e.g. mathematical derivations that may interrupt the flow of your paper's argument should form a separate Appendix section. Do not, however, use appendices to lengthen your article unnecessarily. If the material can be found in another work, cite this work rather than reproduce it.

%=========================================== BIBLIOGRAPHY =========================================%
% nocite is used to provide examples for other types of documents:
\nocite{smith_2007_the,brown_2012_the,jones_2006_the,hodges_2004_the,harrison_2005_the,iet_2013_report}
\nocite{brown_2004_the,smith_1925,abbott_2005_the,standard_2006}
\bibliographystyle{ietperso}
\bibliography{ietperso}

%============================================ TEXCOUNT ============================================%
%TC:ignore
% Uncomment to use texcount macro
%\immediate\write18{texcount \jobname.tex -inc -incbib -sum -html -out=\jobname.html}
%TC:endignore

\end{ietbody}
\end{document}